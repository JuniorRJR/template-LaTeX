
%%%%%%%%%%%%%%%%%%%%%%%%%%%%%%%%%%%%%%%%%%%%%%%%%%
%%                                              %%
%%          Importação de pacotes               %%
%%                                              %%
%%%%%%%%%%%%%%%%%%%%%%%%%%%%%%%%%%%%%%%%%%%%%%%%%%

% Acrônimos (Siglas e Abreviaturas)
\usepackage{acronym}

% Pacote para o desenvolvimento de algoritmos e codificação utf8
\usepackage[portuguese,lined,boxed,ruled]{algorithm2e}
\usepackage{algorithmic}

% Fontes e símbolos matemáticos
\usepackage{amsfonts, amsmath, amssymb}

\usepackage[brazil]{babel}

\usepackage{booktabs}


% Manutenção das legendas em imagens (fonte pequena, 10pt)
\usepackage[font=small]{caption}

% Manutenção da marcação em listas (enumerator)
\usepackage{enumitem}

% Codificação da fonte em 8 bits
\usepackage[T1]{fontenc}

% Inserir figuras
\usepackage{graphicx}

% Dimensões do documento
\usepackage[left=3.0cm, right=2.0cm, top=3.0cm, bottom=2.0cm]{geometry}

% Glossário
\usepackage[nonumberlist,style=index]{glossaries}

% Hifenização das palavras 
\usepackage{hyphenat}

% Índice
\usepackage{imakeidx}


% Identar do primeiro parágrafo de cada seção
\usepackage{indentfirst}

% Acentuação direta
\usepackage[utf8]{inputenc}

% para melhorias de justificação
\usepackage{microtype}

% Mescla de células em tabelas
\usepackage{multirow}

\usepackage{placeins}

% Rotacionar elementos
\usepackage{rotating}

% Manutenção do espaçamento entre linhas
\usepackage{setspace}

% Cria rodapé em tabelas 
\usepackage{threeparttable}

% Texto em Times New Roman
\usepackage{times}

%\usepackage{tocloft}

%%%%%%%%%%%%%%%%%%%%%%%%%%%%
%% Pacotes com Hierarquia %%
%%%%%%%%%%%%%%%%%%%%%%%%%%%%

\usepackage[brazilian, hyperpageref]{backref}	 % Paginas com as citações na bibl

% Referências
\usepackage[alf, bibjustif, abnt-etal-list=0, abnt-etal-text=it]{abntex2cite}



% ---
% Configurações do pacote backref
% Usado sem a opção hyperpageref de backref
\renewcommand{\backrefpagesname}{Citado na(s) página(s):~}
% Texto padrão antes do número das páginas
\renewcommand{\backref}{}
% Define os textos da citação
\renewcommand*{\backrefalt}[4]{
	\ifcase #1 %
		Nenhuma citação no texto.%
	\or
		Citado na página #2.%
	\else
		Citado #3 vezes nas páginas #2.%
	\fi}%
% ---